\documentclass[a4paper,12pt]{article}
\usepackage[utf8]{inputenc}
\usepackage{amsmath, amssymb}
\usepackage{graphicx}
\usepackage{hyperref}

\title{\textbf{Planeación del Proyecto: Secreto Compartido}}
\author{}
\date{\today}

\begin{document}

\maketitle

\section*{Equipo de Trabajo}
\begin{itemize}
    \item \textbf{Scrum Master:} [Nombre del Integrante 1]
    \item \textbf{Desarrollador 1:} [Nombre del Integrante 2]
    \item \textbf{Desarrollador 2:} [Nombre del Integrante 3]
    \item \textbf{Analista/Tester:} [Nombre del Integrante 4]
\end{itemize}

\section*{Duración y Metodología}
\begin{itemize}
    \item \textbf{Tiempo total:} 4 semanas
    \item \textbf{Metodología:} Scrum
    \item \textbf{Sprints:} 2 sprints de 2 semanas
    \item \textbf{Reuniones:} Diarias de 15 min (Daily Stand-Up) y retrospectiva al final de cada sprint
\end{itemize}

\section*{Objetivos del Proyecto}
\begin{itemize}
    \item Implementar un protocolo de comunicación para cálculo conjunto de funciones con datos privados.
    \item Aprender y programar operaciones en el campo primo $\mathbb{Z}_p$, incluyendo suma, multiplicación e inverso modular.
    \item Implementar y evaluar la interpolación de Lagrange en $\mathbb{Z}_p$ para la reconstrucción de datos secretos.
    \item Simular una red P2P y definir la estructura de los mensajes para la comunicación entre nodos.
    \item Medir tiempos de cómputo y evaluar el uso de la red en la implementación del protocolo.
\end{itemize}

\section*{Plan de Trabajo (Backlog)}
\subsection*{Sprint 1 (Semana 1-2)}
\begin{itemize}
    \item \checkmark Definir estructura del proyecto y entorno de desarrollo.
    \item \checkmark Implementar generación y reparto de secretos.
    \item \checkmark Simular red P2P y definir estructura de mensajes.
    \item \checkmark Interpolación de Lagrange para reconstrucción.
\end{itemize}

\subsection*{Sprint 2 (Semana 3-4)}
\begin{itemize}
    \item \checkmark Implementar medidas de seguridad (SSL en fase final).
    \item \checkmark Optimización con primos de Mersenne.
    \item \checkmark Medición de tiempos y pruebas de concepto.
    \item \checkmark Documentación y preparación de la entrega.
\end{itemize}

\section*{Entregables}
\begin{itemize}
    \item Código funcional con implementación del protocolo.
    \item Informe técnico con metodología, resultados y pruebas.
    \item Presentación final del proyecto.
\end{itemize}

\section*{Herramientas y Tecnologías}
\begin{itemize}
    \item \textbf{Lenguaje:} Python / Java
    \item \textbf{Gestión de código:} GitHub
    \item \textbf{Gestión ágil:} Trello / Jira
    \item \textbf{Criptografía:} OpenSSL, NumPy
\end{itemize}

\end{document}

